%===============================================================================
\chapter{Proper orthogonal decomposition}\label{ch03}
%===============================================================================

The present chapter intoduces the concept of proper orthogonal decomposition (POD), a reduce-order modelling tool use to provide a statistical description of coherent structures.
Section~\ref{ch03:s1} provides a historical view of the use of this tool in the area of fluid mechanics, and defines its mathematical foundation.
$\S$~\ref{ch03:s2} present the discrete formulation that will be used later in Chapter~\ref{ch04} in combination of neural network to reconstruct turbulent flow fields.
Finally, $\S$~\ref{ch03:s3} present the extended POD, a linear-relation tool used to reconstruct turbulent flow fields.

%-------------------------------------------------------------------------------
\section{Historical perspective and mathematical framework}\label{ch03:s1}
%-------------------------------------------------------------------------------
Principal component analysis in a statistical tool develop in the works of \citet{pearson1901liii} and \citet{hotelling1933analysis} that decomposes any signal as the linear combination of a set of orthonormal basis, allowing to capture hidden pattern in large datasets.
Introduced in the area of fluid mechanics by \citet{lumley1967structure} under the name of POD, it allows to decompose a fluctuating velocity field $\boldsymbol{u}(\boldsymbol{x}, t)$ as a linear combination of orthonormal basis, composed of spatially orthonormal functions $\boldsymbol{\phi}_i(\boldsymbol{x})$ times a temporal orthonormal basis:
\begin{equation}
  \boldsymbol{u}(\boldsymbol{x}, t) \approx \sum_{i=1}^{N_m} \psi_i(t) \sigma_i \boldsymbol{\phi}_i(\boldsymbol{x}),
  \label{ch03:eq01}
\end{equation}

\noindent with $N_m$ being the number of POD modes, $\sigma_i$ weighting the contribution of each mode to each instantaneous field and $\psi_i(t)$ referring to the time coefficients.
Note that when $N_m$ tends to infinity, equation~\ref{ch03:eq01} becomes a equality.

In the original work of \citet{lumley1967structure}, POD aims to maximize the projection of the velocity fluctuations $\boldsymbol{u}(\boldsymbol{x}, t)$ on their spatial basis $\boldsymbol{\phi}_i(\boldsymbol{x})$ that belong to a general Hilbert space $\mathcal{H}$:
\begin{equation}
  \max_{\boldsymbol{\phi}_i(\boldsymbol{x})\epsilon\mathcal{H}}\biggl\langle\int_{\Omega} \bigl(\boldsymbol{u}(\boldsymbol{x}, t), \boldsymbol{\phi}_i(\boldsymbol{x})\bigl)d\boldsymbol{x}\biggl\rangle,
  \label{ch03:eq02}
\end{equation}

\noindent where $\bigl(\cdot,\cdot\bigl)$ indicates a scalar product, $\Omega$ refers to the entire observation domain and $\bigl\langle\cdot,\cdot\bigl\rangle$ indicates am ensemble average.
The choice of basis functions $\boldsymbol{\phi}_i(\boldsymbol{x})$ affects the solution of equation. In the case of POD, this set of basis function is chosen to be orthonormal i.e., fulfilling the orthogonality constraint
\begin{equation}
  \biggl\langle\int_{\Omega} \bigl(\boldsymbol{\phi}_i(\boldsymbol{x}), \boldsymbol{\phi}_j(\boldsymbol{x})\bigl)d\boldsymbol{x}\biggl\rangle=\delta_{ij},
  \label{ch03:eq03}
\end{equation}

\noindent where $\delta_{ij}$ is the Kronecker delta.
\citet{holmes2012turbulence} shows that the problem defined by equations~\ref{ch03:eq02}-\ref{ch03:eq03} is equivalent to find the solution for Fredholm equation:
\begin{equation}
  \biggl\langle\int_{\Omega} \bigl(\mathcal{R}(\boldsymbol{x},\boldsymbol{x}^{\prime}), \boldsymbol{\phi}_i(\boldsymbol{x^{\prime}})\bigl)d\boldsymbol{x^{\prime}}\biggl\rangle=\sigma_i^2\boldsymbol{\phi}_i(\boldsymbol{x}),
  \label{ch03:eq04}
\end{equation}

\noindent where $\mathcal{R}(\boldsymbol{x},\boldsymbol{x}^{\prime})$ is the two-point spatial correlation.
Since any combination of the first $r$ POD modes composes an orthonormal basis that minimizes the Frobenius norm of the flow field, POD provides flow field modes ordered by their energy content.

An alternative approach to compute the POD modes of a set of instantaneoeus turbulent velocity fields is the \textit{snapshot method} suggested by \citet{sirovich1987turbulence}.
This method proposes to project $\boldsymbol{u}(\boldsymbol{x}, t)$ on their temporal basis $\psi_i(t)$, such that:
\begin{equation}
  \max_{\psi_i(t)\epsilon\mathcal{H}}\biggl\langle\int_{\mathcal{T}} \bigl(\boldsymbol{u}(\boldsymbol{x}, t), \psi_i(t)\bigl)dt\biggl\rangle,
  \label{ch03:eq05}
\end{equation}

\noindent where $\mathcal{T}$ refers to the entire temporal domain.
Equation~\ref{ch03:eq04}, together with the orthogonality condition for $\psi_i(t)$, defined as:
\begin{equation}
  \biggl\langle\int_{\mathcal{T}} \bigl(\psi_i(t), \psi_i(t)\bigl)dt\biggl\rangle=\delta_{ij},
  \label{ch03:eq06}
\end{equation}

\noindent compose a problem defined by the Fredholm equation such that:
\begin{equation}
  \biggl\langle\int_{\Omega} \bigl(\mathcal{R}_t(t,t^{\prime}), \psi_i(t^{\prime})\bigl)dt^{\prime}\biggl\rangle=\sigma_i^2\psi_i(t),
  \label{ch03:eq07}
\end{equation}

\noindent where $\mathcal{R}_t(t,t^{\prime})$ is the two-point temporal correlation.
This approach is significantly useful when the available temporal information is significantly smaller than the spation one.

Apart from fluid mechanics, POD has been extensively applied in many research areas, such as data compression \citep{andrews1967adaptive}, signal analysis \citep{algazi1969optimality}, oceanography \citep{preisendorfer1988principal} or astrophysics \citep{soummer2012detection}, to name a few.
With respect to fluid mechanics, Chapter~\ref{ch02} has reviewed the application of POD as a tool to study coherent structures in wall-bounded flows, but there are other problems where it has been successfully applied.
For example, \citet{mendez2017pod} proposed a method to remove the background from PIV images, which is especially useful for eliminating static reflections that occur when the PIV laser hits an object.
Nonetheless, the classical application of the method is the data decomposition of flow fields, for which several examples can be found.
One of the first was the data decomposition of a turbulent channel flow carried out by \citet{moin1989characteristic}.
Since then, POD has been applied in several flow topologies, such as the cylinder wake \citep{deane1991low,feng2011proper,raiola2016wake}, turbulent jets \citep{glauser1991dynamics,berry2017application,mendez2019multi}, flapping wings \citep{liang2015symmetry,troshin2018modeling,raiola2021data}, wall-mounted obstacles \citep{mallor2018wall,mallor2019modal,guemes2019flow}, or mixing layers \citep{delville1999examination,citriniti2000reconstruction,kaiser2014cluster} among many others \citep{rowley2017model}.
It is worth to note that many of these works have also used the POD to study the coherent structures present in turbulent shear flows.

%-------------------------------------------------------------------------------
\section{Discrete formulation of proper orthogonal decomposition}\label{ch03:s2}
%-------------------------------------------------------------------------------

Commonly, the data acquired from experiments or numerical simulations come on discrte form i.e., $N_t$ samples are acquired on a spatil grid composed of $N_p$ grid points, therefore requiring a discrete formulation.
When using the \textit{snapshot method}, the three fluctuating velocity components of each instantaneous field are rearranged into a snapshot matrix $U$ such as:
\begin{equation}
    \mathbf{U}=\begin{bmatrix}
    u_{x_1}^{t_1} & \dots  & u_{x_{N_p}}^{t_1} & v_{x_1}^{t_1} & \dots  & v_{x_{N_p}}^{t_1} & w_{x_1}^{t_1} & \dots  & w_{x_{N_p}}^{t_1}\\
    \vdots         & \ddots & \vdots         & \vdots         & \ddots & \vdots  & \vdots         & \ddots & \vdots \\
    u_{x_1}^{t_{N_t}} & \dots  & u_{x_{N_p}}^{t_{N_t}} & v_{x_1}^{t_{N_t}} & \dots  & v_{x_{N_p}}^{t_{N_t}}  & w_{x_1}^{t_{N_t}} & \dots  & w_{x_{N_p}}^{t_{N_t}}
    \end{bmatrix}.
    \label{ch03:eq08}
\end{equation}

Then, POD spatial modes can be evaluated solving the eigenvalue problem of the spatial correlation matrix $S$ as follows:
\begin{equation}
    \mathbf{S}=\mathbf{U}^T\mathbf{U}=\boldsymbol{\Phi}^T\boldsymbol{\Lambda}\boldsymbol{\Phi},
    \label{ch03:eq09}
\end{equation}

\noindent where $\boldsymbol{\Phi}$ is a matrix whose rows contain the spatial POD modes, while $\boldsymbol{\Lambda}$ is a diagonal matrix with elements $\lambda_i=\sigma_i^2$, which represent the variance content of each mode.
Note that the singular value decomposition of the spatial correlation matrix $S$ shown in equation~\ref{ch03:eq09} is the discrete version of the Fredholm equation presented in equation~\ref{ch03:eq04}.
Considering that equation~\ref{ch03:eq01} can be expressed in matrix form as:
\begin{equation}
  \boldsymbol{U} =  \boldsymbol{\Psi}\boldsymbol{\Sigma}\boldsymbol{\Phi},
  \label{ch03:eq10}
\end{equation}

\noindent where $\Sigma$ is a diagonal matrix containing the square root of the mode variances $\sigma_i$, the POD coefficients $\psi_i(t)$ are obtained by projecting the flow fields on the spatial POD modes computed with equation~\ref{ch03:eq09} as follows:
\begin{equation}
  \mathbf{\Psi} =  \mathbf{S} \mathbf{\Phi}^T\mathbf{\Sigma^{-1}}.
  \label{ch03:eq11}
\end{equation}

In the same manner as equation~\ref{ch03:eq07}, the discrete approach can be formulated to solve the temporal correlation matrix:
\begin{equation}
    \mathbf{T}=\mathbf{U}\mathbf{U}^T=\boldsymbol{\Psi}\boldsymbol{\Lambda}\boldsymbol{\Psi}^T,
    \label{ch03:eq12}
\end{equation}

\noindent where the ratio $N_t/N_p$ conditions the choice between this approach $(N_t/N_p>1)$ or \textit{snapshot method} $(N_t/N_p<1)$.

%-------------------------------------------------------------------------------
\section{Extended proper orthogonal decomposition}\label{ch03:s3}
%-------------------------------------------------------------------------------

Extended proper orthogonal decomposition (EPOD), originally suggested by \citet{maurel2001extended}, is a statistical tool that allows to draw the correlation between the flow features extracted from two different dataset. For instance, given a signal $\boldsymbol{w}(\boldsymbol{x}, t)$ containing the pressure an shear stress fluctuations at the wall, it is possible to calculate its extended POD modes as its projection onto the temporal basis defined for $\boldsymbol{u}(\boldsymbol{x}, t)$ as:
\begin{equation}
  \sigma_i^e\boldsymbol{\phi}_i^e(\boldsymbol{x})=\biggl\langle\int_{\Omega} \bigl(\psi_i(t),\boldsymbol{w}(\boldsymbol{x}, t)\bigl)d\boldsymbol{x}\biggl\rangle,
  \label{ch03:eq13}
\end{equation}

\noindent where superindex $e$ stands for \textit{extended}.
The strength of equation~\ref{ch03:eq13} is that allows to extract the correlation between synchronized measurements through the extended POD modes, being possible to quantify the relationship between the coherent structures populating each one of them.

Rearranging the wall information contained in $\boldsymbol{w}(\boldsymbol{x}, t)$ in a matrix form:
\begin{equation}
  \mathbf{W}=\begin{bmatrix}
      \tau_{w_{x_{x_1}}}^{t_1} & \dots  & \tau_{w_{x_{N_p}}}^{t_1} & \tau_{w_{z_{x_1}}}^{t_1} & \dots  & \tau_{w_{z_{N_p}}}^{t_1} & p_{w_{x_1}}^{t_1} & \dots  & p_{w_{N_p}}^{t_1}\\
      \vdots         & \ddots & \vdots         & \vdots         & \ddots & \vdots  & \vdots         & \ddots & \vdots \\
      \tau_{w_{x_{x_1}}}^{t_{N_t}} & \dots  & \tau_{w_{x_{N_p}}}^{t_{N_t}} & \tau_{w_{z_{x_1}}}^{t_{N_t}} & \dots  & \tau_{w_{z_{N_p}}}^{t_{N_t}}  & p_{w_{x_1}}^{t_{N_t}} & \dots  & p_{w_{N_p}}^{t_{N_t}}
      \end{bmatrix},
  \label{ch03:eq14}
\end{equation}

\noindent and considering that $\mathbf{W}$ can be decomposed analysis:
\begin{equation}
  \boldsymbol{W} =  \boldsymbol{\Psi_w}\boldsymbol{\Sigma_w}\boldsymbol{\Phi_w},
  \label{ch03:eq15}
\end{equation}

\noindent it is possible to express equation~\ref{ch03:eq13} in discrete form analysis as:
\begin{equation}
    \boldsymbol{\Sigma}^e\boldsymbol{\Phi}^e=\boldsymbol{\Psi}\boldsymbol{W}=\boldsymbol{\Psi}\boldsymbol{\Psi_w}\boldsymbol{\Sigma_w}\boldsymbol{\Phi_w}=\boldsymbol{\Xi}\boldsymbol{\Sigma_w}\boldsymbol{\Phi_w},
    \label{ch03:eq16}
\end{equation}

\noindent where matrix $\boldsymbol{\Xi}$ contains the temporal correlation between both quantities.

Assuming that the POD for both quantities has reached statistical convergence, it is possible to estimate any of the quantities through a linear projection on the extended POD modes analysis:
\begin{equation}
  \boldsymbol{U}^{\dagger} =  \boldsymbol{\Xi}\boldsymbol{\Psi_w}\boldsymbol{\Sigma}\boldsymbol{\Phi},
  \label{ch03:eq15}
\end{equation}

\noindent where superscript $\dagger$ refers to reconstructed quantities.
\citet{discetti2018estimation} showed that the energy-optimality of POD modes mentioned in $\S$~\ref{ch03:s1} can then be enforced by employing statistical filters on the temporal correlation matrix.
Apart from the examples related to wall turbulence presented in Chapter~\ref{ch02}, EPOD has been showed to provide accurate predictions in other types of flows, such as the far-field predictions of turbulent jets carried out by \citet{tinney2008low}.

One of the most important properties of EPOD is its connection to the linear stochastic estimation developed by \citep{adrian1979conditional}.
\citet{boree2003extended} proved mathematically that LSE and EPOD are equivalent methods.
Following \citet{holmes2012turbulence}, this relation can be resumed in two main properties.
First, suitable averages of LSE events produce the POD modes and second, all LSE events are linear combinations of POD modes.
